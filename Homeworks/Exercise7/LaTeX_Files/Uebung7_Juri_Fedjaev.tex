\documentclass[
a4paper, 
12pt, 
]{article}

\usepackage[ngerman,english]{babel}			% Kapitel/Chapter. typographic rules
%\usepackage[german]{babel}



\usepackage[utf8]{inputenc}			% Encoding with umlauts and ß
%\usepackage{lipsum}				 			% testing text as \lipsum[1-3] 			 
%\usepackage{titling}							% imports \theauthor
\usepackage{graphicx}						% include graphics
\usepackage{siunitx}							% corretct formatting of units
\usepackage[framed,numbered,autolinebreaks,useliterate]{mcode} % for matlab code

\usepackage{amsmath}            % nice equations
\usepackage{url}                			% URLs
%\usepackage{natbib}             		% author-year bibliography style
\usepackage{hyperref}           	% PDF links
%\usepackage{subfig}             	% Subfigures (a), (b), etc
%\usepackage{nomencl}            	% Nomenclature	
\usepackage{tcolorbox}
\usepackage{Systemtheorie}		 	% style with headers/footers/logo/firstpage
\fancyfoot[R]{Juri Fedjaev} 
\usepackage{units}		% nice fractions using \nicefrac
\usepackage{blindtext}
\usepackage{mcode}
%\usepackage[scale=0.9]{geometry}
\usepackage[update,prepend]{epstopdf}

 
% ----------------------------------------------------------------------------


\begin{document}
	
	\thispagestyle{firstpage} 			% use different style here (from .sty file)
	
	\section*{Neuroprothetik -- Übung 7: CI Signalverarbeitung}
	\subsection{Aufgabe 1}
	\subsection*{a)} (Implementieren Sie die Filterbank mit Butterworth IIR Filtern und Ploten sie die Frequenzgänge der Filter für 3 und 22 Kanäle.)\\
	\begin{figure}[h]
\centering
\includegraphics[width=0.6\linewidth]{Plots/a22_ampSpektrum}
\caption{Frequenzgang des gesprochenen Wortes 'Neuroprothetik', aufgenommen mit einer Samplingrate von $f_s = 44.1$ kHz.}
\label{fig:a22_ampSpektrum}
\end{figure}

	\begin{figure}[h]
		\centering
		\includegraphics[width=0.6\linewidth]{Plots/a3_fbSpektrum}
		\caption{Frequenzgänge der Filter für 3 Kanäle.}
		\label{fig:a3_fbSpektrum}
	\end{figure}

	
	\begin{figure}
\centering
\includegraphics[width=0.6\linewidth]{Plots/a22_fbSpektrum}
\caption{Frequenzgänge der Filter für 22 Kanäle.}
\label{fig:a22_fbSpektrum}
\end{figure}

\subsection*{b)} (Nehmen Sie ein beliebiges Wort mit ihrer Soundkarte auf und verarbeiten Sie es mit
ihrer Filterbank. Ploten Sie die Ausgänge der jeweiligen Kanäle für eine 6 und eine 12
Kanal Filterbank.)
\begin{figure}[h]
\centering
\includegraphics[width=0.6\linewidth]{Plots/b6}
\caption{Gesprochenes Wort 'Neuroprothetik' gefiltert mit einer 6-kanaligen Filterbank}
\label{fig:b6}
\end{figure}


\begin{figure}[h]
\centering
\includegraphics[width=0.6\linewidth]{Plots/b12}
\caption{Gesprochenes Wort 'Neuroprothetik', gefiltert mit einer 12-kanaligen Filterbank}
\label{fig:b12}
\end{figure}

\newpage
\subsection*{c)} (Fügen sie die Kanalausgänge nach der Filterung wieder zusammen. Plotten sie das
Langzeitspektren und das Spectrogram (Kurzzeitspektren) eines beliebigen aufgenommenen
Wortes, vor und nach der Filterung, mit einer 3 und einer 12 Kanal Filterbank.)

\begin{figure}[h]
\centering
\includegraphics[width=0.95\linewidth]{Plots/c3}
\caption{3-Kanal Filterbank: Plot zu gesprochenem Wort 'Neuroprothetik' (oben link), gefilterter und wieder zusammengesetzte Version des Wortes (oben rechts), Spektrogramm vor Filterung (Mitte) und nach Filterung (unten).}
\label{fig:c3}
\end{figure}

\begin{figure}[h]
\centering
\includegraphics[width=0.95\linewidth]{Plots/c12}
\caption{12-Kanal Filterbank: Plot zu gesprochenem Wort 'Neuroprothetik' (oben link), gefilterter und wieder zusammengesetzte Version des Wortes (oben rechts), Spektrogramm vor Filterung (Mitte) und nach Filterung (unten).}
\label{fig:c12}
\end{figure}


	
	

\end{document}