\documentclass[
a4paper, 
12pt, 
]{article}

% \usepackage[ngerman,english]{babel}			% Kapitel/Chapter. typographic rules
\usepackage[english]{babel}



\usepackage[utf8]{inputenc}			% Encoding with umlauts and ß
%\usepackage{lipsum}				 			% testing text as \lipsum[1-3] 			 
%\usepackage{titling}							% imports \theauthor
\usepackage{graphicx}						% include graphics
\usepackage{siunitx}							% corretct formatting of units
\usepackage[framed,numbered,autolinebreaks,useliterate]{mcode} % for matlab code

\usepackage{amsmath}            % nice equations
\usepackage{url}                			% URLs
%\usepackage{natbib}             		% author-year bibliography style
\usepackage{hyperref}           	% PDF links
%\usepackage{subfig}             	% Subfigures (a), (b), etc
%\usepackage{nomencl}            	% Nomenclature	
\usepackage{tcolorbox}
\usepackage{Systemtheorie}		 	% style with headers/footers/logo/firstpage
\fancyfoot[R]{Juri Fedjaev} 
\usepackage{units}		% nice fractions using \nicefrac
\usepackage{blindtext}
\usepackage{mcode}
%\usepackage[scale=0.9]{geometry}
\usepackage[update,prepend]{epstopdf}

 
% ----------------------------------------------------------------------------


\begin{document}
	
	\thispagestyle{firstpage} 			% use different style here (from .sty file)
	
	\section*{Neuroprothetik -- Exercise 5: Multicompartment Model}
	\subsection{Create a Multicompartment Model}
	\emph{See code in \mcode{function V = MultiCompartement(i_stimulus)}. Input is the current vector $i_{stimulus}$}.
	
	\subsection{Experiments}
	Each of the following simulations were run for $100$ ms with time steps of $\Delta t = 25~\mu$s at $6.3~$ \textcelsius. 
	\subsubsection{Experiment 1}
	\begin{figure}[h]
\centering
\includegraphics[width=0.8\linewidth]{Plots/Plot2_1}
\caption{Visualization of the action potential propagation in an axon when stimulated at the first compartment with a rectangular $5$ ms pulse with an amplitude of $10$ \nicefrac{$\mu A$}{$cm^2$}.}
\label{fig:Plot2_1}
\end{figure}
	\subsubsection{Experiment 2}
	When artificially stimulated with an input current at a specific compartment, both of the neighboring compartments are receptive (\emph{not} refractory). Therefore, the action potential (AP) propagates in both directions along the axon (see fig.~\ref{fig:Plot2_2}). Now, whenever two APs intersect from opposing directions on the axon, both of the neighboring compartments of the intersection are in their refractory period and are thus not able to make the APs propagate any further. This is the case in the visualization below (fig.~\ref{fig:Plot2_2}) in approx. compartment 50. 
	\begin{figure}[h]
		\centering
		\includegraphics[width=0.8\linewidth]{Plots/Plot2_2}
		\caption{Visualization of the action potential propagation in an axon when stimulated simultaneously at compartments 20 and 80 with a rectangular $5$ ms pulse with an amplitude of $10$ \nicefrac{$\mu A$}{$cm^2$}.}
		\label{fig:Plot2_2}
	\end{figure}
	
	\subsubsection{Impact of the different parameters of the model}
	As seen in the lecture, the parameter that has major impact on the speed of action potential propagation is the axon resistance $R_A$. When multiplying the axon resistance with a constant scaling factor $r_{scale}$, we get the following plots:
	\begin{figure}[h]
\centering
\includegraphics[width=0.8\linewidth]{Plots/Plot3_1}
\caption{Scaling factor $r_{scale} = 1.5$, the intersection of the two APs is now at approx. $95$ ms (cf. fig.~\ref{fig:Plot2_2}, approx. $60$ ms). }
\label{fig:Plot3_1}
\end{figure}
	\begin{figure}[h]
		\centering
		\includegraphics[width=0.8\linewidth]{Plots/Plot3_2}
		\caption{Scaling factor $r_{scale} = 2$, the axon resistance is too high to let the action potential propagate at all (current in the next compartment not high enough for triggering an AP).}
		\label{fig:Plot3_2}
	\end{figure}


	


	

	
	

\end{document}